\documentclass[a4paper]{article}
\usepackage{algorithm}
\usepackage{algpseudocode}
\usepackage{frpseudocode}
\begin{document}
\begin{algorithm}
\caption{Algorithme d'Euclide}
\begin{algorithmic}[1]
\Function{Euclide}{$a, b$}
\Comment{PGCD de a et b}
\State $r\gets a\bmod b$
\While{$r\not=0$}\Comment{Si r = 0, on a la réponse}
\State $a\gets b$
\State $b\gets r$
\State $r\gets a\bmod b$
\EndWhile
\State \Return $b$\Comment{Le PGCD est b}
\EndFunction
\end{algorithmic}
\end{algorithm}

\begin{algorithm}
\caption{Algorithme d'Euclide}
\begin{algorithmic}[1]
\Procedure{AfficheurMultiple}{$str$}
\Comment{Démo boucle pour}
\State $r\gets a\bmod b$
\For{$i$}{$0$}{$15$}
\State Afficher $str$ \Comment{On affiche la chaîne}
\EndFor
\EndProcedure
\end{algorithmic}
\end{algorithm}

\end{document}